\chapter{Introduction}
\section{Project Motivation and Problem Statement}
Motorcycle accidents are a major contributor to traffic fatalities worldwide, particularly in regions with a high volume of two-wheeler traffic. A significant factor in the severity of these accidents is the non-usage of safety helmets. Manual monitoring of helmet compliance by law enforcement is labor-intensive and not scalable. Therefore, an automated system that can detect helmet law violations can serve as a crucial tool for promoting road safety and aiding enforcement efforts.

The primary problem is to develop a system that is not only accurate but also cost-effective and capable of running in real-time on accessible hardware. High-end GPU-based systems are often too expensive for widespread deployment. This project aims to bridge that gap by leveraging a lightweight deep learning model on a Raspberry Pi.

\section{Project Objectives}
The main objectives of this project are as follows:
\begin{itemize}
    \item To collect and prepare a suitable dataset for training a multi-class object detector capable of identifying motorbikes, helmets, and license plates.
    \item To train a YOLOv8n object detection model on the prepared dataset.
    \item To optimize the trained model for deployment on an edge device by converting it to the TensorFlow Lite (TFLite) format.
    \item To deploy the optimized model on a Raspberry Pi 3B+ and develop an inference script to perform detection on images.
    \item To evaluate the performance of the system in terms of both detection accuracy and inference speed on the target hardware.
\end{itemize}

\section{Scope of the Project}
The scope of this project is focused on detection from static images. While the system is designed with real-time performance in mind, the implementation will be demonstrated on single image files. The project covers the complete pipeline from data collection, model training, optimization, and deployment, culminating in a functional proof-of-concept on the Raspberry Pi. It does not extend to real-time video stream processing or integration with a larger traffic management infrastructure.
