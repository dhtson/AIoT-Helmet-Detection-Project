\chapter{Implementation}
This chapter details the key scripts developed for the project.

\section{Data Preparation Script (\texttt{download.py})}
This script automates the downloading and merging of datasets from Roboflow.
\begin{lstlisting}[language=Python, caption={Key snippet from download.py.}]
import os
import yaml
from roboflow import Roboflow

# Configuration for datasets and API key
ROBOFLOW_API_KEY = "YOUR_ROBOFLOW_API_KEY"
DATASET_URLS = {
    "motobike": "cdio-zmfmj/motobike-detection",
    "helmet_license": "cdio-zmfmj/helmet-lincense-plate-detection-gevlq"
}
# ... (rest of the script) ...

# Merging datasets
final_class_list = sorted(list(all_classes), key=lambda x: class_map[x])

final_yaml_content = {
    'path': os.path.abspath(COMBINED_DIR),
    'train': 'images/train',
    'val': 'images/valid',
    'test': 'images/test',
    'names': {i: name for i, name in enumerate(final_class_list)}
}

final_yaml_path = os.path.join(COMBINED_DIR, "data.yaml")
with open(final_yaml_path, 'w') as f:
    yaml.dump(final_yaml_content, f, sort_keys=False, indent=4)

# ... (rest of the script) ...
\end{lstlisting}

\section{Model Training Script (\texttt{train.py})}
This script handles the training of the YOLOv8n model and the final export to TFLite.
\begin{lstlisting}[language=Python, caption={Key snippet from train.py.}]
from ultralytics import YOLO

# Configuration for training
DATA_YAML_PATH = "./datasets/combined_dataset/data.yaml"
EPOCHS = 100
PATIENCE = 5
IMG_SIZE = 320

# Load a pre-trained model
model = YOLO('yolov8n.pt')

# Train the model
results = model.train(
    data=DATA_YAML_PATH,
    epochs=EPOCHS,
    patience=PATIENCE,
    imgsz=IMG_SIZE,
)

# Export the best model to TFLite format
best_model = YOLO(results.save_dir / 'weights' / 'best.pt')
best_model.export(format='tflite')
\end{lstlisting}

\section{Inference Script (\texttt{detect.py})}
This is the final script deployed on the Raspberry Pi to perform detection on an image.
\begin{lstlisting}[language=Python, caption={Key snippet from the Raspberry Pi inference script.}]
import cv2
import tflite_runtime.interpreter as tflite
import numpy as np

# Load the TFLite model and allocate tensors
interpreter = tflite.Interpreter(model_path="best_float16.tflite")
interpreter.allocate_tensors()

# Get input and output tensors.
input_details = interpreter.get_input_details()
output_details = interpreter.get_output_details()

# ... (Image preprocessing) ...

# Run inference
interpreter.set_tensor(input_details[0]['index'], input_data)
interpreter.invoke()
output_data = interpreter.get_tensor(output_details[0]['index'])

# ... (Post-processing and drawing boxes) ...
\end{lstlisting}
